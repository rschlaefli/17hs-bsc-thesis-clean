\begin{zusammenfassung}
Der Indische Sommermonsun ist von grosser Bedeutung für über eine Milliarde Menschen. Die Wichtigkeit einer präzisen statistischen Auswertung des Monsunverhaltens wird dadurch deutlich. In dieser Arbeit analysieren wir die geographische Verteilung von extremem Monsunregenfall und entwickeln eine neue Methode zur Vorhersage des Monsunbeginns. Wir berechnen Netzwerke aus korrelierten Orten auf dem Indischen Subkontinent und analysieren diese mit verbreiteten Indikatoren der Netzwerkzentralität. Dadurch zeigt sich eine relative Wichtigkeit von Regionen wie dem Indischen Ozean, dem Tibetischen Plateau und Nordpakistan. Basierend auf aktuellen Methoden im Bereich der Neuronalen Netzwerke entwickeln wir ausserdem ein neues Modell zur Vorhersage des Monsunbeginns basierend auf räumlich-zeitlichen Wetterdaten. Mit darauf aufbauenden Experimenten zeigen wir, dass unser Modell den Monsunanfang mehrere Tage im Voraus genauer als bestehende Methoden vorhersagen kann.
\end{zusammenfassung}
