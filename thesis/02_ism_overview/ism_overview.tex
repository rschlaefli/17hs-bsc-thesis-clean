\chapter{An Overview of the Indian Summer Monsoon}
\label{c:ism_overview}
The Indian Summer Monsoon is a global climatic phenomenon that has been actively researched for decades. Even though many theories and hypotheses exist, the factors influencing its strength, timing and variability have still not been fully identified and understood. This chapter aims to provide a short overview of the observed behavior and current scientific understanding of ISM, as far as the foreknowledge might prove useful for the further parts of this work.

\section{Seasons \& Progression}
\label{st:ism_seasons}
The ISM progresses over the Indian subcontinent in three main phases: the pre-monsoon season (March-May), the primary monsoon season (June-September) and the post-monsoon season (October-December) \citep{Stolbova.2015}.

The months of the pre-monsoon season correspond to the Indian summer and are characterized by high temperatures with low amounts of rainfall over regions in the Himalayas and south-west India. Many predictions about the further development of ISM are commonly based on the pre-monsoon season, especially on the patterns that occur leading up to the primary monsoon season \citep{Stolbova.2015}.

A sudden transition with rapidly increasing amounts of daily rainfall, moisture, and kinetic energy, as well as increased vertical shear in horizontal winds\footnote{A gradient between wind speeds at different heights or pressure levels.}, then marks the beginning of the primary monsoon season \citep{Pradhan.2017}. This transition, the "onset" of monsoon, occurs at different times for different parts of the Indian subcontinent, as can be seen in \cref{fig:onset_propagation}. First arriving in Kerala on the southwestern tip of the subcontinent around the first of June, the ISM then propagates north- and northwestwards, where it reaches the Pakistani border around the 15th of July \citep{Willetts.2017}.

\begin{figure}[h]
  \centering
  [TODO: image for onset propagation (use stolbova?)]
  \caption{Propagation of monsoon over the Indian subcontinent.}
  \label{fig:onset_propagation}
\end{figure}

After the ISM has fully covered the Indian subcontinent in July, most places experience strong weather conditions well into ...

[TODO: describe withdrawal of monsoon and its timing]

[TODO: describe post-monsoon phase and the northeast monsoon]

\section{Main Drivers}
\label{st:ism_factors}
In its most basic form, the ISM can be seen as a sea breeze of enormous extent. The hot summer weather during the pre-monsoon season causes a heating of the Indian subcontinent. An increasingly large temperature gradient due to slower heating of the Indian Ocean then causes air to flow onto the subcontinent, bringing with it the moisture necessary to cause precipitation \citep{Willetts.2017}.

Intense heating and heat dissipation of the high-altitude Tibetan Plateau lead to an increase in the tropospheric temperature, creating an area of low-pressure near the surface that attracts moisture from surrounding areas over the Indian Ocean [TODO: cite stolbova and secondary]. The low-pressure zone further causes strong vertical air currents from south of the Tibetan Plateau, aiding the ISM with its propagation into the far north of India \citep{Pradhan.2017} [TODO: cite li\&yanai or so].

On a larger scale, the ISM has been linked to several parts of the global atmospheric circulatory system. A shift of the subtropical jet stream to the north of the Tibetan Plateau [TODO: cite yin?] as well as interaction with the westerly jet stream have been found to influence the behavior of the ISM \citep{Ordonez.2016, Stolbova.2015}. A strong westerly jet in the lower troposphere over Kerala seems to generally correlate with the monsoon onset of the region \citep{Ordonez.2016} [TODO: cite rao?]. Additionally, the Somali jet passing over the Arabian sea cools down the body of water, further strengthening the temperature gradient \citep{Stolbova.2015}.

The trade winds of the northern and southern hemisphere meet to create the Intertropical Convergence Zone (ITCZ), a belt of low-pressure close to the thermal aequator. When the ITCZ moves north following the summer months, it further increases the variability of weather events and thus reduces their predictability \citep{Stolbova.2015}.

A further factor in the development and strength of the ISM is the El Niño Southern Oscillation (ENSO). The warming of the Humboldt ocean current during El Niño years causes the land-ocean temperature gradient to decrease. This tends to decrease the amount of rainfall the Indian subcontinent receives and can delay the onset of monsoon by several days \citep{Pradhan.2017, Willetts.2017}. Conversely, the cooling of the current during La Niña can result in more rainfall and an earlier onset due to a higher temperature gradient.

A combination of the above factors creates a low-pressure channel (a "monsoon through") closely south of and parallel to the Tibetan Plateau that serves as a main source of moisture during the ISM \citep{Stolbova.2015}.

This summarization of the impacting factors of the ISM is not exhaustive, as the ISM is much more complex and some of its behavior has still not been fully explained. However, knowledge of these basic factors should already provide a good intuition for \cref{c:event_sync} and \cref{c:part2}.

\section{Social Impact}
\label{st:ism_impact}
As we have already seen, the Indian Summer Monsoon is one of the most impactful large-scale meteorological events on earth. It affects the lives of up to one-fourth of the world's population: the people living on and around the Indian subcontinent \citep{Stolbova.2015}. The heavy rainfalls that the ISM brings with it are of great importance for the population in India and surrounding countries.

Livelihood on the Indian subcontinent is strongly coupled to a timely occurrence of the ISM. Monsoon rainfalls are responsible for 80 percent of the annual precipitation on the Indian subcontinent \citep{Jin.2017}. Farmers depend on these rainfalls to water their crops and feed their livestock. Up to 2012, more than half of a year's rice harvest was still grown during the monsoon period \citep{Auffhammer.2012}.

The agricultural sector accounts for almost one-fifth of the India's GDP, and about 50\% of the population in India either directly or indirectly depend on the agricultural sector \cite{.05.01.2018}. Monsoon rainfalls further provide the majority of drinking water in many parts of India \citep{Stolbova.2015}.

The impactfulness of ISM makes clear the need of being able to predict the monsoon behavior and its onset and withdrawal dates accurately. If farmers knew in advance when monsoon rainfalls would start, they could wait until the appropriate time to plant their crops. On the contrary, if they get surprised by either early or late onset of monsoon, their crop yield might be reduced, or their crops destroyed entirely.

\section{Trends \& Outlook}
\label{st:ism_trends}
Much like the global climate, the ISM is exposed to climate change and other trends that can impact its variability and predictabily. A major trend that was prevalent during the second half of the 20th century was a decrease in precipitation over northern-central India (close to the Tibetan Plateau) \citep{Jin.2017}.

[TODO: cite Jin 6,7] propose that such a drying trend could be connected to a warming of the surrounding Indian Ocean. A resulting weakening of the land-ocean temperature gradient could have been responsible for the lower rainfall amounts. A further hypothesis states that large-scale deforestation could have decreased the amount of transpiration from plants, resulting in less moisture and thus precipitation [TODO: cite Jin 8].

[TODO: reversal of drying trend]

[TODO: general climate change]

[TODO: more?]





